\documentclass{article}
\usepackage{eecstex}
\usepackage{physics}
\usepackage{pgfplots}

\renewcommand{\thesubsection}{\thesection.\arabic{subsection}}
\renewcommand{\thesubsubsection}{\thesubsection.\alph{subsubsection}}
\renewcommand{\labelenumi}{\arabic{enumi}.}
\newcommand{\F}{\mathcal{F}}

\title{EE 120 HW 06}
\author{Bryan Ngo}
\date{2021-03-05}

\begin{document}

\maketitle

\section{Continuous Time Fourier Series}

\begin{equation}
    x(t) = \sum_{k \in \Z} a_k e^{j \omega_0 k t} = \sum_{k \in \Z} a_k e^{j \frac{2\pi}{T} kt}
\end{equation}

\subsection{}

\begin{theorem}
    If \(x(t) = x(-t)\) for all \(t\), then \(a_k \in \R\).
\end{theorem}
\begin{proof}
    By definition of the DTFS coefficients,
    \begin{align}
        a_k &= \frac{1}{T} \int_{\langle T \rangle} x(t) e^{-j \omega_0 k t} \, dt \\
        &= \frac{1}{T} \int_{-\frac{T}{2}}^{\frac{T}{2}} x(t) e^{-j \omega_0 k t} \, dt = \frac{2}{T} \int_0^{\frac{T}{2}} x(t) e^{-j \omega_0 k t} \, dt \\
        &= \frac{1}{T} \int_0^{\frac{T}{2}} x(t) e^{j \omega_0 k t} \, dt + \underbrace{\frac{1}{T} \int_0^{\frac{T}{2}} x(t) e^{-j \omega_0 k t} \, dt}_{\text{by definition of even function}} \\
        &= \frac{1}{T} \int_0^{\frac{T}{2}} x(t) (e^{j \omega_0 k t} + e^{-j \omega_0 k t}) \, dt = \frac{2}{T} \int_0^{\frac{T}{2}} x(t) \cos(\omega_0 k t) \, dt \in \R
    \end{align}
\end{proof}

\subsection{}

\begin{theorem}
    If \(x(t) = -x(-t)\) for all \(t\), then \(\Re\{a_k\} = 0\).
\end{theorem}
\begin{proof}
    By definition of the DTFS coefficients,
    \begin{align}
        a_k &= \frac{1}{T} \int_{\langle T \rangle} x(t) e^{-j \omega_0 k t} \, dt = \frac{1}{T} \int_{-\frac{T}{2}}^{\frac{T}{2}} x(t) e^{-j \omega_0 k t} \, dt \\
        &= \frac{1}{T} \int_{-\frac{T}{2}}^0 x(t) e^{-j \omega_0 k t} \, dt + \frac{1}{T} \int_0^{\frac{T}{2}} x(t) e^{-j \omega_0 k t} \, dt \\
        &= \underbrace{-\frac{1}{T} \int_0^{\frac{T}{2}} x(t) e^{j \omega_0 k t} \, dt}_{\text{by definition of odd function}} + \frac{1}{T} \int_0^{\frac{T}{2}} x(t) e^{-j \omega_0 k t} \, dt \\
        &= -\frac{1}{T} \int_0^{\frac{T}{2}} x(t) (e^{j \omega_0 k t} - e^{-j \omega_0 k t}) \, dt = -\frac{2j}{T} \int_0^{\frac{T}{2}} x(t) \sin(\omega_0 k t) \, dt \implies \Re\{a_k\} = 0
    \end{align}
\end{proof}

\subsection{}

\begin{theorem}
    If \(x(t) = -x\qty(t + \frac{T}{2})\) for all \(t\), then \(a_{2k} = 0\).
\end{theorem}
\begin{proof}
    By definition of the DTFS coefficients,
    \begin{align}
        a_k &= \frac{1}{T} \int_{\langle T \rangle} x(t) e^{-j \omega_0 k t} \, dt \\
        &= \frac{1}{T} \int_{-\frac{T}{2}}^{\frac{T}{2}} x(t) e^{-j \omega_0 k t} \, dt \\
        &= \frac{1}{T} \int_{-\frac{T}{2}}^0 x(t) e^{-j \omega_0 k t} \, dt + \frac{1}{T} \int_0^{\frac{T}{2}} x(t) e^{-j \omega_0 k t} \, dt
    \end{align}
    Letting \(t = u + \frac{T}{2} \implies du = dt\),
    \begin{align}
        a_k &= \frac{1}{T} \int_0^{\frac{T}{2}} x\qty(t + \frac{T}{2}) e^{-j \omega_0 k \qty(t + \frac{T}{2})} \, dt + \frac{1}{T} \int_0^{\frac{T}{2}} x(t) e^{-j \omega_0 k t} \, dt \\
        &= \frac{1}{T} \int_0^{\frac{T}{2}} -x(t) e^{-j \omega_0 k t} e^{-j \omega_0 k \frac{T}{2}} + x(t) e^{-j \omega_0 k t} \, dt
    \end{align}
    For even \(k\), we get \(e^{-j \frac{2\pi}{T} 2m \frac{T}{2}} = 1\) as the extra multiplicative term for \(m \in \Z\), so the integral evaluates to \(0\).
\end{proof}

\section{DTFS}

\begin{align}
    x[n + 4m] &= x[n] \\
    \sum_{n \in [-1, 2]} x[n] &= 2 \\
    \sum_{n \in [-1, 2]} (-1)^n x[n] &= 4 \\
    \sum_{n \in [-1, 2]} x[n] \cos\qty[\frac{\pi}{2} n] &= \sum_{n \in [-1, 2]} x[n] \sin\qty[\frac{\pi}{2} n] = 0
\end{align}

\subsection{}

From the first property, we know that \(N_0 = 4\) at most.
By definition of the DTFS,
\begin{equation}
    a_k = \frac{1}{4} \sum_{n \in [-1, 2]} x[n] e^{-j \frac{\pi}{2} k n} = \frac{1}{4} (x[-1] e^{j \frac{\pi}{2} k} + x[0] + x[1] e^{-j \frac{\pi}{2} k} + x[2] e^{-j \pi k})
\end{equation}
Then from the last property,
\begin{align}
    \frac{1}{2} \sum_{n \in [-1, 2]} x[n] (e^{j \frac{\pi}{2}} n + e^{-j \frac{\pi}{2} n}) &= 0 \\
    \frac{1}{2j} \sum_{n \in [-1, 2]} x[n] (e^{j \frac{\pi}{2} n} - e^{-j \frac{\pi}{2} n}) &= 0
\end{align}
which tells us that \(a_1 = a_{-1} = 0\) by pattern-matching.
Plugging \(k = 0\) into the DTFS and using the second property tells us that \(a_0 = \frac{1}{4} \sum_{n \in [-1, 2]} x[n] = \frac{1}{2}\).
Likewise, plugging in \(k = 2\) into the DTFS and using the third property tells us that \(a_2 = \frac{1}{4} \sum_{n \in [-1, 2]} (-1)^n x[n] = 1\).
Thus, we have
\begin{equation}
    x[n] = \sum_{k \in [-1, 2]} a_k e^{j \frac{\pi}{2} k n} = \frac{1}{2} + e^{j \pi n} = \frac{1}{2} + (-1)^n
\end{equation}
\begin{center}
    \begin{tikzpicture}
        \begin{axis}[
            xlabel=\(n\), ylabel={\(x[n]\)},
            axis lines=middle,
            ymin=-1, ymax=2
        ]
        \addplot[
            ycomb,
            color=blue,
            mark=*
        ]
        coordinates {
            (-1, -1/2)
            (0, 3/2)
            (1, -1/2)
            (2, 3/2)
        };
        \end{axis}
    \end{tikzpicture}
\end{center}

\subsection{}

From the plot, we find that \(N_0 = 2\) so \(\omega_0 = \pi\).
Finding the DTFS of \(x[n]\),
\begin{align}
    a_k &= \frac{1}{2} \sum_{n \in [-1, 0]} x[n] e^{-j \pi k n} = -\frac{1}{4} (-1)^k + \frac{3}{4} \\
    x[n] &= \sum_{k \in [-1, 0]} a_k e^{j \pi k n} = e^{j \pi n} + \frac{1}{2} e^{j \pi (0) n}
\end{align}
Note that we have a complex exponential at harmonic frequencies \(\omega = \pi\) and \(\omega = 0\) and coefficients \(1\) and \(\frac{1}{2}\), respectively.
These are the only frequencies present in our signal, so our DTFT must resemble two spikes at the respective frequencies, appropriately scaled.
Then, we can use the fact that the DTFT of any \(x[n]\) represented by the DTFS as
\begin{equation}
    X(e^{j \omega}) = \sum_{k \in [-1, 0]} a_k \delta(\omega - \pi k) = \delta(\omega + \pi) + \frac{1}{2} \delta(\omega)
\end{equation}

\end{document}

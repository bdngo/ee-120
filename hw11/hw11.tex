\documentclass{article}
\usepackage{eecstex}
\usepackage{pgfplots}
\usepackage{physics}

\DeclareMathOperator{\Shah}{Shah}

\renewcommand{\thesubsection}{\thesection.\arabic{subsection}}
\renewcommand{\thesubsubsection}{\thesubsection.\alph{subsubsection}}
\renewcommand{\labelenumi}{\arabic{enumi}.}
\newcommand{\F}{\mathcal{F}}
\newcommand{\sinc}{\operatorname{sinc}}
\newcommand{\rect}{\operatorname{rect}}


\title{EE 120 HW 11}
\author{Bryan Ngo}
\date{2021-04-15}

\begin{document}

\maketitle

\section{Zero-Padding the DFT}

\subsection{}

\begin{equation}
    X(e^{j \omega}) = \sum_{n \in \Z} x[n] e^{-j \omega n}
\end{equation}

\subsection{}

\begin{equation}
    X_k = \sum_{n \in [0, N - 1]} x[n] e^{-j k \frac{2\pi}{N} n}
\end{equation}

\subsection{}

Finding the sampled DTFT,
\begin{align}
    X_p(e^{j \omega}) = X_p[k] &= X(e^{j \omega}) \sum_{k \in \Z} \delta\qty(\omega - k\frac{2\pi}{N}) = X(e^{j k \frac{2\pi}{N}}) \\
    &= \sum_{n \in \Z} x[n] e^{-j k \frac{2\pi}{N} n} = \sum_{n \in [0, N - 1]} x[n] e^{-j k \frac{2\pi}{N} n} = X_{k \bmod N}
\end{align}

\subsection{}

\begin{align}
    \hat{x}[n] &= \frac{2M + N}{N} \F^{-1}_{2M + N}\{\hat{X}_m\} \\
    &= \frac{\cancel{2M + N}}{N} \frac{1}{\cancel{2M + N}} \sum_{n \in [-M, M + N - 1]} \hat{X}[k] e^{j k \frac{2\pi}{2M + N}} \\
    &= \frac{1}{N} \sum_{n \in [-M, M + N - 1]} \hat{X}[k] e^{j k \frac{2\pi}{2M + N} n} \\
    &= \frac{1}{N} \sum_{n \in [0, N - 1]} \hat{X}[k] e^{j k \frac{2\pi}{2M + N}n}
\end{align}
Where we can adjust the indices due to the zero padding.
On the right hand side, we have
\begin{equation}
    x_c\qty(\frac{N}{2M + N} n) = \frac{1}{N} \sum_{n \in [0, N - 1]} X[k] \exp\qty(j k \frac{2\pi}{\cancel{N}} \frac{\cancel{N}}{2M + N} n) = \frac{1}{N} \sum_{n \in [0, N - 1]} X_c[k] e^{j k \frac{2\pi}{2M + N}}
\end{equation}
Thus, we have proven them equal along the interval \([0, N - 1]\).

\subsection{}

We can pattern match and find that \(T = \frac{N}{2M + N} < 1\).
This means that \(\hat{x}[n]\) is sampled at a higher rate than \(x[n]\).

\section{First Order LCCDE System}

\begin{equation}
    \dv{y_c(t)}{t} + y_c(t) = x_c(t)
\end{equation}

\subsection{}

Plugging in \(e^{j \omega}\) into \(G\) and using the derivative property of the CTFT,
\begin{equation}
    j \omega Y_c(\omega) + Y_c(\omega) = e^{j \omega} \implies Y_c(\omega) = \frac{e^{j \omega}}{1 + e^{j \omega}}
\end{equation}
Then, we can deconvolve \(Y_c(\omega) = H(\omega) e^{j \omega}\),
\begin{align}
    h(t) &= \F\qty{\frac{Y_c(\omega)}{e^{j \omega}}} = \F\qty{\frac{1}{1 + j \omega}} = e^{-t} u(t) \\
    y_p(t) &= \sum_{n \geqslant 0} e^{-t} \delta(t - nT) \implies y[n] = e^{-nT} u[n]
\end{align}

\subsection{}

Taking the DTFT of \(y[n]\) using the time scaling property and since
\begin{equation}
    e^{-n} u[n] \longleftrightarrow \frac{1}{1 - e^{-1 - j \omega}}
\end{equation}
We have
\begin{equation}
    e^{-nT} u[n] \longleftrightarrow \frac{1}{T (1 - e^{-1 - j \frac{\omega}{T}})} = Y(e^{j \omega})
\end{equation}
On order to ensure that \(w[n] = \delta[n]\),
\begin{align}
    H(e^{j \omega}) Y(e^{j \omega}) &= W(e^{j \omega}) = 1 \\
    h[n] &= \F^{-1}\qty{T - \frac{T}{e}e^{-j \frac{\omega}{T}}} = \delta[n] - \frac{T}{e} \delta\qty[n - \frac{1}{T}]
\end{align}

\section{Sample in Frequency Domain}

\subsection{}

\begin{center}
    \begin{tikzpicture}
        \begin{axis}[
            xlabel=\(\omega\), ylabel={\(|X_p(\omega)|\) (\(\omega_s = 1\))},
            axis lines=middle,
            ymin=0, ymax=2
        ]
        \addplot[
            ycomb,
            color=blue,
            mark=*
        ]
        coordinates {
            (-2, 0.5)
            (-1, 0.9)
            (0, 2)
            (1, 0.9)
            (2, 0.5)
        };
        \end{axis}
    \end{tikzpicture}
\end{center}

\subsection{}

On the left hand side,
\begin{align}
    X_p(t) &= \sum_{n \in \Z} X(\omega - n\omega_s) = X(\omega) P(\omega)\\
    x_p(t) &= x(t) \ast p(t) = x(t) \ast \frac{1}{\omega_s} \sum_{k \in \Z} \delta\qty(t - n\frac{2\pi}{\omega_s}) = \frac{1}{\omega_s} \sum_{n \in \Z} x\qty(t - n\frac{2\pi}{\omega_s})
\end{align}
Where we take the inverse FT of \(P(\omega)\) through duality.
This means that \(x_p(t)\) is an infinite sum of shifted versions of \(x(t)\).

\subsection{}

We must sample in the frequency domain such that \(\omega_s \geqslant \frac{1}{2T}\), or that the spacing must be at least half than the period width of the time-domain signal.

\subsection{}

We multiply \(x_p(t)\) with a low-pass brickwall filter with cutoff period \(T\) to recover \(x(t)\).
In the frequency domain, this is equivalent to convolution with a sinc function.

\end{document}

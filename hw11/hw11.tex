\documentclass{article}
\usepackage{eecstex}
\usepackage{pgfplots}
\usepackage{physics}

\DeclareMathOperator{\Shah}{Shah}

\renewcommand{\thesubsection}{\thesection.\arabic{subsection}}
\renewcommand{\thesubsubsection}{\thesubsection.\alph{subsubsection}}
\renewcommand{\labelenumi}{\arabic{enumi}.}
\newcommand{\F}{\mathcal{F}}
\newcommand{\sinc}{\operatorname{sinc}}
\newcommand{\rect}{\operatorname{rect}}


\title{EE 120 HW 11}
\author{Bryan Ngo}
\date{2021-04-15}

\begin{document}

\maketitle

\section{Zero-Padding the DFT}

\subsection{}

\begin{equation}
    X(e^{j \omega}) = \sum_{n \in \Z} x[n] e^{-j \omega n}
\end{equation}

\subsection{}

\begin{equation}
    X_k = \sum_{n \in [0, N - 1]} x[n] e^{-j k \frac{2\pi}{N} n}
\end{equation}

\subsection{}

\begin{align}
    X_p[k] = \qty(\sum_{n \in \Z} x[n] e^{-j \omega n}) \qty(\sum_{m \in \Z} \delta\qty(\omega - m \frac{2\pi}{N}))
    &= \sum_{n \in \Z} \sum_{m \in \Z} x[n] e^{-j (\omega - m\frac{2\pi}{N}) n}
\end{align}
Meaning that a frequency domain sampled DTFT is simply 

\subsection{}

\subsection{}

\section{First Order LCCDE System}

\begin{equation}
    \dv{y_c(t)}{t} + y_c(t) = x_c(t)
\end{equation}

\subsection{}

The step response is \(y_s(t) = (1 - e^{-t}) u(t)\).
Then, given the fact that the delta function is the derivative of the unit step function,
\begin{align}
    h(t) &= \F\qty{\frac{Y_c(\omega)}{e^{j \omega}}} = \F\qty{\frac{1}{1 + j \omega}} = e^{-t} u(t) \\
    y_p(t) &= \sum_{n \geqslant 0} e^{-t} \delta(t - nT) \implies y[n] = e^{-nT} u[nT]
\end{align}

\subsection{}

Taking the DTFT of \(y[n]\) using the time scaling property and since
\begin{equation}
    e^{-n} u[n] \longleftrightarrow \frac{1}{1 - e^{-1 - j \omega}}
\end{equation}
We have
\begin{equation}
    e^{-nT} u[nT] \longleftrightarrow \frac{1}{T (1 - e^{-1 - j \frac{\omega}{T}})}
\end{equation}
On order to ensure that \(w[n] = \delta[n]\),
\begin{align}
    H(e^{j \omega}) Y(e^{j \omega}) &= W(e^{j \omega}) = 1 \\
    h[n] &= \F^{-1}\qty{T - \frac{T}{e}e^{-j \frac{\omega}{T}}} \\
    &= \delta[n] - \frac{T}{e} \delta\qty[n - \frac{1}{T}]
\end{align}

\section{Sample in Frequency Domain}

\begin{equation}
    P(\omega) = \sum_{n \in \Z} \delta(\omega - n\omega_s)
\end{equation}

\subsection{}

\begin{center}
    \begin{tikzpicture}
        \begin{axis}[
            xlabel=\(\omega\), ylabel={\(|X_p(\omega)|\) (\(\omega_s = 1\))},
            axis lines=middle,
            ymin=0, ymax=2
        ]
        \addplot[
            ycomb,
            color=blue,
            mark=*
        ]
        coordinates {
            (-2, 0.5)
            (-1, 0.9)
            (0, 2)
            (1, 0.9)
            (2, 0.5)
        };
        \end{axis}
    \end{tikzpicture}
\end{center}

\subsection{}

\begin{align}
    X_p(t) = \sum_{n \in \Z} X(\omega - n\omega_s) \\
    x_p(t) = \frac{2\pi}{\omega_s} \sum_{n \in \Z} x\qty(t + \frac{2\pi n}{\omega_s})
\end{align}
This means that \(x_p(t)\) is an infinite sum of shifted versions of \(x(t)\).

\subsection{}

We must sample in the frequency domain such that \(\omega_s \geqslant \frac{1}{2T}\), or that the spacing must be at least than the period width of the time-domain signal.

\subsection{}

We multiply \(x_p(t)\) with a low-pass rectangular filter to recover \(x(t)\).

\end{document}

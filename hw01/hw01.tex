\documentclass{article}
\usepackage{eecstex}
\usepackage{pgfplots}

\renewcommand{\thesection}{\Alph{section}}
\renewcommand{\thesubsection}{\thesection.\arabic{subsection}}

\title{EE 120 HW 01}
\author{Bryan Ngo}
\date{2021-01-22}

\begin{document}

\maketitle

\section{Euler's Formula}

\begin{equation}
    e^{i \theta} = \cos(\theta) + i \sin(\theta)
\end{equation}

\subsection{}

\begin{theorem}
    \begin{equation}
        \cos(\theta) = \frac{e^{i \theta} + e^{-i \theta}}{2}
    \end{equation}
\end{theorem}
\begin{proof}
    \begin{align}
        e^{-i \theta} &= \cos(-\theta) + i \sin(-\theta) \\
        &= \cos(\theta) - i \sin(\theta) \\
        &\Rightarrow e^{i \theta} + e^{-i \theta} = 2 \cos(\theta) \\
        &\Rightarrow \cos(\theta) = \frac{e^{i \theta} + e^{-i \theta}}{2}
    \end{align}
\end{proof}
\begin{theorem}
    \begin{equation}
        \sin(\theta) = \frac{e^{i \theta} - e^{-i \theta}}{2}
    \end{equation}
\end{theorem}
\begin{proof}
    \begin{align}
        e^{-i \theta} &= \cos(\theta) - i \sin(\theta) \\
        &\Rightarrow e^{i \theta} - e^{-i \theta} = 2 \sin(\theta) \\
        &\Rightarrow \sin(\theta) = \frac{e^{i \theta} - e^{-i \theta}}{2}
    \end{align}
\end{proof}

\subsection{}

\begin{theorem}
    \begin{equation}
        (\cos(\theta) + i \sin(\theta))^n = \cos(n \theta) + i \sin(n \theta)
    \end{equation}
\end{theorem}
\begin{proof}
    \begin{equation} 
        (\cos(\theta) + i \sin(\theta))^n = (e^{i \theta})^n = e^{i n \theta} = \cos(n \theta) + i \sin(n \theta)
    \end{equation}
\end{proof}

\subsection{}

\begin{theorem}
    \begin{equation}
        \sum_{k \in [1, N]} A_k \cos(\omega t + \phi_k) = A \cos(\omega t + \phi)
    \end{equation}
\end{theorem}
\begin{proof}
    \begin{align}
        \sum_{k \in [1, N]} A_k \cos(\omega t + \phi_k) &= \sum_{k \in [1, N]} A_k \frac{e^{i (\omega t + \phi_k)} + e^{-i (\omega t + \phi_k)}}{2} \\
        &= \sum_{k \in [1, N]} A_k e^{i \phi_k} \frac{e^{i \omega t} + e^{-i \omega t}}{2} \\
        &= \cos(\omega t) \sum_{k \in [1, N]} A_k e^{i \phi_k} \\
        &= \cos(\omega t) A e^{i \phi_k} = A \cos(\omega t + \phi_k)
    \end{align}
\end{proof}

\section{Periodicity of Signals}

\begin{enumerate}
    \item Yes, \(T = \frac{\pi}{2}\)
    \item Yes, \(T = 2\)
    \item No
    \item Yes, \(T = 4 \pi\)
    \item No
    \item No
    \item Yes, \(T = \frac{14}{6} = \frac{7}{3}\)
\end{enumerate}

\section{Signal Transformation}

\begin{tikzpicture}
    \begin{axis}[
        grid=both,
        xmin=-3, xmax=7,
        ymin=-1, ymax=3,
        xtick={0, 3, 5},
        ytick={0, 2},
        axis lines=middle,
        xlabel=\(x\), ylabel=\(y\),
        title=\(x(t)\)
    ]
    \addplot[
        color=blue,
        mark=none,
    ]
    coordinates {
        (-2, 0)
        (0, 0)
        (0, 2)
        (3, 0)
        (3, 2)
        (5, 0)
        (7, 0)
    };
    \end{axis}
\end{tikzpicture}

\subsection{}

\begin{tikzpicture}
    \begin{axis}[
        grid=both,
        xmin=-7, xmax=3,
        ymin=-1, ymax=3,
        xtick={0, -3, -5},
        ytick={0, 2},
        axis lines=middle,
        xlabel=\(x\), ylabel=\(y\),
        title=\(x(-t)\)
    ]
    \addplot[
        color=blue,
        mark=none,
    ]
    coordinates {
        (2, 0)
        (0, 0)
        (0, 2)
        (-3, 0)
        (-3, 2)
        (-5, 0)
        (-7, 0)
    };
    \end{axis}
\end{tikzpicture}

\subsection{}

\begin{tikzpicture}
    \begin{axis}[
        grid=both,
        xmin=-3, xmax=15,
        ymin=-1, ymax=3,
        xtick={0, 6, 10},
        ytick={0, 2},
        axis lines=middle,
        xlabel=\(x\), ylabel=\(y\),
        title=\(x(2t)\)
    ]
    \addplot[
        color=blue,
        mark=none,
    ]
    coordinates {
        (-2, 0)
        (0, 0)
        (0, 2)
        (6, 0)
        (6, 2)
        (10, 0)
        (14, 0)
    };
    \end{axis}
\end{tikzpicture}

\subsection{}

\begin{tikzpicture}
    \begin{axis}[
        grid=both,
        xmin=-3, xmax=10,
        ymin=-1, ymax=3,
        xtick={2, 5, 7},
        ytick={0, 2},
        axis lines=middle,
        xlabel=\(x\), ylabel=\(y\),
        title=\(x(t + 2)\)
    ]
    \addplot[
        color=blue,
        mark=none,
    ]
    coordinates {
        (-2, 0)
        (2, 0)
        (2, 2)
        (5, 0)
        (5, 2)
        (7, 0)
        (9, 0)
    };
    \end{axis}
\end{tikzpicture}

\subsection{}

\begin{tikzpicture}
    \begin{axis}[
        grid=both,
        xmin=-3, xmax=7,
        ymin=-1, ymax=3,
        xtick={-1, 0.5, 1.5},
        ytick={0, 2},
        axis lines=middle,
        xlabel=\(x\), ylabel=\(y\),
        title=\(x\left(\frac{t}{2} - 1\right)\)
    ]
    \addplot[
        color=blue,
        mark=none,
    ]
    coordinates {
        (-2, 0)
        (-1, 0)
        (-1, 2)
        (0.5, 0)
        (0.5, 2)
        (1.5, 0)
        (7, 0)
    };
    \end{axis}
\end{tikzpicture}

\subsection{}

\begin{tikzpicture}
    \begin{axis}[
        grid=both,
        xmin=-15, xmax=5,
        ymin=-1, ymax=3,
        xtick={1, -8, -14},
        ytick={0, 2},
        axis lines=middle,
        xlabel=\(x\), ylabel=\(y\),
        title=\(x\left(\frac{t}{2} - 1\right)\)
    ]
    \addplot[
        color=blue,
        mark=none,
    ]
    coordinates {
        (5, 0)
        (1, 0)
        (1, 2)
        (-8, 0)
        (-8, 2)
        (-14, 0)
        (-15, 0)
    };
    \end{axis}
\end{tikzpicture}

\section{Integral Review}

\subsection{}

\begin{equation}
    \int_{-1}^\infty e^{-2t} \, dt = \left.-\frac{1}{2} e^{-2t}\right|_{-1}^\infty = -\frac{1}{2} \lim_{R \to \infty} (\cancelto{0}{e^{-2R}} - e^2) = \frac{e^2}{2}
\end{equation}

\subsection{}

\begin{align}
    y(t) &= \int_{-\infty}^\infty u(\tau) e^{-u(t - \tau)} u(t - \tau) \, d\tau \\
    &= \int_0^\infty e^{-u(t - \tau)} u(t - \tau) \, d\tau
\end{align}

\end{document}

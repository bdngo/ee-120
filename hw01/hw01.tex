\documentclass{article}
\usepackage{eecstex}

\title{EE 120 HW 01}
\author{Bryan Ngo}
\date{2021-01-22}

\begin{document}

\maketitle

\section{Euler's Formula}

\begin{equation}
    e^{i \theta} = \cos(\theta) + i \sin(\theta)
\end{equation}

\subsection{}

\begin{theorem}
    \begin{equation}
        \cos(\theta) = \frac{e^{i \theta} + e^{-i \theta}}{2}
    \end{equation}
\end{theorem}
\begin{proof}
    \begin{align}
        e^{-i \theta} &= \cos(-\theta) + i \sin(-\theta) \\
        &= \cos(\theta) - i \sin(\theta) \\
        &\Rightarrow e^{i \theta} + e^{-i \theta} = 2 \cos(\theta) \\
        &\Rightarrow \cos(\theta) = \frac{e^{i \theta} + e^{-i \theta}}{2}
    \end{align}
\end{proof}
\begin{theorem}
    \begin{equation}
        \sin(\theta) = \frac{e^{i \theta} - e^{-i \theta}}{2}
    \end{equation}
\end{theorem}
\begin{proof}
    \begin{align}
        e^{-i \theta} &= \cos(\theta) - i \sin(\theta) \\
        &\Rightarrow e^{i \theta} - e^{-i \theta} = 2 \sin(\theta) \\
        &\Rightarrow \sin(\theta) = \frac{e^{i \theta} - e^{-i \theta}}{2}
    \end{align}
\end{proof}

\subsection{}

\begin{theorem}
    \begin{equation}
        (\cos(\theta) + i \sin(\theta))^n = \cos(n \theta) + i \sin(n \theta)
    \end{equation}
\end{theorem}
\begin{proof}
    \begin{equation} 
        (\cos(\theta) + i \sin(\theta))^n = (e^{i \theta})^n = e^{i n \theta} = \cos(n \theta) + i \sin(n \theta)
    \end{equation}
\end{proof}

\subsection{}

\begin{theorem}
    \begin{equation}
        \sum_{k \in [1, N]} A_k \cos(\omega t + \phi_k) = A \cos(\omega t + \phi)
    \end{equation}
\end{theorem}
\begin{proof}
    \begin{align}
        \sum_{k \in [1, N]} A_k \cos(\omega t + \phi_k) &= \sum_{k \in [1, N]} A_k \frac{e^{i (\omega t + \phi_k)} + e^{-i (\omega t + \phi_k)}}{2} \\
        &= \sum_{k \in [1, N]} A_k e^{i \phi_k} \frac{e^{i \omega t} + e^{-i \omega t}}{2} \\
        &= \cos(\omega t) \sum_{k \in [1, N]} A_k e^{i \phi_k} \\
        &= \cos(\omega t) A e^{i \phi_k} = A \cos(\omega t + \phi_k)
    \end{align}
\end{proof}

\section{Periodicity of Signals}

\begin{enumerate}
    \item Yes, \(T = \frac{\pi}{2}\)
    \item Yes, \(T = 2\)
    \item No
    \item Yes, \(T = 4 \pi\)
\end{enumerate}

\section{Signal Transformation}



\end{document}

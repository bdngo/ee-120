\documentclass{article}
\usepackage{eecstex}
\usepackage{physics}

\renewcommand{\thesubsection}{\thesection.\arabic{subsection}}
\renewcommand{\thesubsubsection}{\thesubsection.\alph{subsubsection}}
\renewcommand{\labelenumi}{\arabic{enumi}.}
\newcommand{\F}{\mathcal{F}}
\newcommand{\sinc}{\operatorname{sinc}}
\newcommand{\rect}{\operatorname{rect}}


\title{EE 120 HW 08}
\author{Bryan Ngo}
\date{2021-03-16}

\begin{document}

\maketitle

\section{Separable Function}

\begin{equation}
    \begin{array}{|c|c|}
        \hline
        \text{Signal} & \text{Separation} \\
        \hline
        \frac{1}{x^2 - y^2} & \emptyset \\
        \frac{1}{x^2 + 4y^2 - 4xy} & \emptyset \\
        \frac{1}{xy - y + 2x - 2} & \frac{1}{x - 1} \frac{1}{y + 2} \\
        e^{-\frac{1}{2} (x^2 + y^2)} & e^{-\frac{x^2}{2}} e^{-\frac{y^2}{2}} \\
        \sin(x + y) + \sin(x - y) & 2 \sin(x) \cos(y) \\
        \begin{cases}
            1 & |x| < x_0, |y| < y_0 \\
            0 & \text{otherwise}
        \end{cases} & \Pi\qty(\frac{x}{2x_0}) \Pi\qty(\frac{y}{2y_0}) \\
        \hline
    \end{array}
\end{equation}

\section{All Pass System}

\begin{equation}
    H(e^{j \omega}) = \frac{e^{-j \omega} - a}{1 - ae^{-j \omega}}
\end{equation}

\subsection{}

Note that \(H(e^{j \omega})\) can be rewritten as
\begin{equation}
    H(e^{j \omega}) = \frac{e^{-j \omega} - a}{e^{-j \omega} (e^{j \omega} - a)} = e^{j \omega} \frac{e^{-j \omega} - a}{e^{j \omega} - a}
\end{equation}
Since the numerator and the denominators are conjugates, \(|z| = |z^\ast|\), and \(|e^{j \omega}| = 1\), we have \(|H(e^{j \omega})| = 1\).

\subsection{}

Note that \(\angle H(e^{j \omega})\) can be rewritten as
\begin{align}
    H(e^{j \omega}) &= \frac{e^{-j \omega} - a}{e^{-j \omega} (e^{j \omega} - a)} = e^{j \omega} \frac{e^{-j \omega} - a}{e^{j \omega} - a} \\
    \angle H(e^{j \omega}) &= \omega + \angle(e^{-j \omega} - a) - \angle(e^{j \omega} - a) \\
    &= \omega + \angle((\cos(\omega) - a) - j \sin(\omega)) - \angle ((\cos(\omega) - a) + j \sin(\omega)) \\
    &= \omega + \tan[-1](-\frac{\sin(\omega)}{\cos(\omega) - a}) - \tan[-1](\frac{\sin(\omega)}{\cos(\omega) - a}) \\
    &= \omega - 2\tan[-1](\frac{\sin(\omega)}{\cos(\omega) - a})
\end{align}

\subsection{}

First, convert the signal into a sum of exponentials,
\begin{equation}
    x[n] = \frac{1}{2} (e^{j \frac{\pi}{6} n} + e^{-j \frac{\pi}{6} n} + e^{j \pi n} + e^{-j \pi n})
\end{equation}
Then we can plug in every frequency into the transfer function.
Since \(|H(e^{j \omega})| = 1\), we know that there will only be frequency shifts.
The final output will simply be shifted by each respective frequency by the eigenfunction property.
We can calculate that
\begin{align}
    \pm \frac{\pi}{6} - 2\tan[-1](\frac{\sin(\pm \frac{\pi}{6})}{\cos(\pm \frac{\pi}{6}) - \frac{1}{\sqrt{3}}}) &= \mp \frac{\pi}{2} \\
    \pm \pi - 2\tan[-1](\frac{\sin(\pm \pi)}{\cos(\pm \pi) - \frac{1}{\sqrt{3}}}) &= \pm \pi
\end{align}
So the final output is
\begin{equation}
    y[n] = \frac{1}{2} (e^{j \qty(\frac{\pi}{6} n - \frac{\pi}{2})} + e^{-j \qty(\frac{\pi}{6} n - \frac{\pi}{2})} + e^{j \qty(\pi n + \pi)} + e^{-j \qty(\pi n + \pi)}) = \sin\qty[\frac{\pi}{6} n] - \cos\qty[\pi n]
\end{equation}
Since \(\sin\qty[n - \frac{\pi}{2}] = \cos[n]\) and \(\cos[n + \pi] = -\cos[n]\).

\subsection{}

Note that \(H(e^{j \omega}) = \frac{Y(e^{j \omega})}{X(e^{j \omega})}\), so we can rearrange to get
\begin{equation}
    Y(e^{j \omega}) (1 - ae^{-j \omega}) = (e^{-j \omega} - a) X(e^{j \omega})
\end{equation}
Applying the inverse DTFT to both sides and using the linearity and frequency shifting properties, we get
\begin{equation}
    y[n] - ay[n - 1] = x[n - 1] - ax[n]
\end{equation}

\section{DFT and DTFS}

\begin{equation}
    x = \sum_{k \in \langle p \rangle} X_k \phi_k \longleftrightarrow X_k = \frac{\iprod{x}{\phi_k}}{\iprod{\phi_k}{\phi_k}}
\end{equation}

\subsection{}

The general equation is
\begin{equation}
    \iprod{\psi_k}{\psi_\ell} = \sum_{n \in \langle p \rangle} \psi_k[n] \psi_\ell^\ast[n]
\end{equation}
Substituting the definitions of \(\psi\),
\begin{equation}
    \frac{1}{p^2} \sum_{n \in \langle p \rangle} \phi_k[n] \phi_\ell^\ast[n] = \frac{1}{p^2} \sum_{n \in [0, p - 1]} e^{j \omega_0 n (k - \ell)} = \frac{1}{p^2} \frac{1 - e^{j \frac{2\pi}{p} p n (k - \ell)}}{1 - e^{j \omega_0 (k - \ell)}} = \frac{1}{p^2} \frac{1 - e^{j 2\pi n (k - \ell)}}{1 - e^{j \omega_0 (k - \ell)}}
\end{equation}
If \(k = \ell\), we will get a sum of \(p\) ones.
Since \(k - \ell \in \Z\), if \(k \neq \ell\), we will always have \(0\) in the numerator.
In other words,
\begin{equation}
    \iprod{\psi_k}{\psi_\ell} =
    \begin{cases}
        \frac{1}{p} & k = \ell \\
        0 & k \neq \ell
    \end{cases}
\end{equation}

\subsection{}

We can express \(X_k'\) in terms of \(x\) and \(\Psi\) as
\begin{equation}
    X_k' = p \sum_{k \in \langle p \rangle} x[n] \psi_k^\ast[n] = \frac{\iprod{x}{\psi_k}}{\iprod{\psi_k}{\psi_k}}
\end{equation}
Furthermore, by pattern-matching the fact that
\begin{equation}
    x[n] = \sum_{k \in \langle p \rangle} X_k \phi_k[n] = \frac{1}{p} \sum_{k \in \langle p \rangle} X_k' \phi_k[n]
\end{equation}
We get that \(X_k' = p X_k\), or
\begin{equation}
    X_k' = p X_k = p \frac{1}{p} \sum_{k \in \langle p \rangle} x[n] \phi_k^\ast[n] = \sum_{k \in \langle p \rangle} x[n] e^{-j \omega_0 k n}
\end{equation}
which is the DFT analysis equation.
Furthermore, we can plug in the inner product in terms of \(\Psi\),
\begin{equation}
    x[n] = \sum_{k \in \langle p \rangle} X_k' \psi_k[n] = \sum_{k \in \langle p \rangle} p X_k \frac{1}{p} \phi_k[n] = \sum_{k \in \langle p \rangle} X_k e^{j k \omega_0 n}
\end{equation}
which is the DFT synthesis equation.

\end{document}

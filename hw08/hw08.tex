\documentclass{article}
\usepackage{eecstex}
\usepackage{physics}

\renewcommand{\thesubsection}{\thesection.\arabic{subsection}}
\renewcommand{\thesubsubsection}{\thesubsection.\alph{subsubsection}}
\renewcommand{\labelenumi}{\arabic{enumi}.}
\newcommand{\F}{\mathcal{F}}
\newcommand{\sinc}{\operatorname{sinc}}
\newcommand{\rect}{\operatorname{rect}}


\title{EE 120 HW 08}
\author{Bryan Ngo}
\date{2021-03-16}

\begin{document}

\maketitle

\section{Separable Function}

\begin{equation}
    \begin{array}{|c|c|}
        \hline
        \text{Signal} & \text{Separation} \\
        \hline
        \frac{1}{x^2 - y^2} & \emptyset \\
        \frac{1}{x^2 + 4y^2 - 4xy} & \emptyset \\
        \frac{1}{xy - y + 2x - 2} & \emptyset \\
        e^{-\frac{1}{2} (x^2 + y^2)} & e^{-\frac{x^2}{2}} e^{-\frac{y^2}{2}} \\
        \sin(x + y) + \sin(x - y) & 2 \sin(x) \cos(y) \\
        \begin{cases}
            1 & |x| < x_0, |y| < y_0 \\
            0 & \text{otherwise}
        \end{cases} & \Pi\qty(\frac{x}{2x_0}) \Pi\qty(\frac{y}{2y_0}) \\
        \hline
    \end{array}
\end{equation}

\section{All Pass System}

\begin{equation}
    H(e^{j \omega}) = \frac{e^{-j \omega} - a}{1 - ae^{-j \omega}}
\end{equation}

\subsection{}

Note that \(H(e^{j \omega})\) can be rewritten as
\begin{align}
    H(e^{j \omega}) &= \frac{\cos(\omega) - j \sin(\omega) - a}{1 - a \cos(\omega) + a j \sin(\omega)} \\
    \implies |H(e^{j \omega})| &= \sqrt{\frac{(\cos(\omega) - a)^2 + \sin^2(\omega)}{(1 - a \cos(\omega))^2 + a^2 \sin^2(\omega)}} \\
    &= \sqrt{\frac{\cos^2(\omega) - 2a \cos(\omega) + a^2 + \sin^2(\omega)}{1 - 2a \cos(\omega) + a^2 (\cos^2(\omega) + \sin^2(\omega))}} \\
    &= \sqrt{\frac{1 + a^2 - 2a \cos(\omega)}{1 + a^2 - 2a \cos(\omega)}} = 1
\end{align}

\subsection{}

Note that \(\angle H(e^{j \omega})\) can be rewritten as
\begin{align}
    H(e^{-j \omega}) &= \frac{e^{-j \omega} - a}{e^{-j \omega} (e^{j \omega} - a)} = e^{j \omega} \frac{e^{-j \omega} - a}{e^{j \omega} - a}\\
    \angle H(e^{j \omega}) &= \omega + \angle(e^{-j \omega} - a) - \angle(e^{j \omega} - a) \\
    &= \omega + \angle((\cos(\omega) - a) - j \sin(\omega)) - \angle ((\cos(\omega) - a) + j \sin(\omega)) \\
    &= \omega + \tan[-1](-\frac{\sin(\omega)}{\cos(\omega) - a}) - \tan[-1](\frac{\sin(\omega)}{\cos(\omega) - a}) \\
    &= \omega - 2\tan[-1](\frac{\sin(\omega)}{\cos(\omega) - a})
\end{align}

\subsection{}

Plugging in \(a = \frac{1}{\sqrt{3}}\), we have
\begin{equation}
    H(e^{j \omega}) = \frac{e^{-j \omega} - \frac{1}{\sqrt{3}}}{1 - \frac{1}{\sqrt{3}} e^{-j \omega}}
\end{equation}
In order to find \(y[n] = x[n] \ast h[n]\), we can find \(Y(e^{j \omega}) = X(e^{j \omega}) H(e^{j \omega})\).
Finding \(X(e^{j \omega})\),
\begin{equation}
    X(e^{j \omega}) = \pi \sum_{k \in \Z} \qty(\delta\qty(\omega - \frac{\pi}{6} - 2\pi k) + \delta\qty(\omega + \frac{\pi}{6} - 2\pi k) + \delta(\omega - \pi - 2\pi k) + \delta(\omega + \pi - 2\pi k))
\end{equation}
Calculating \(X(e^{j \omega}) H(e^{j \omega})\),
\begin{equation}
    X(e^{j \omega}) H(e^{j \omega}) = 
\end{equation}

\section{DFT and DTFS}

\subsection{}

When \(k = \ell\),
\begin{equation}
    \frac{1}{p^2} \sum_{k \in \langle p \rangle} \phi_k[n] \phi_k^\ast[n] = \frac{1}{p}
\end{equation}
When \(k \neq \ell\),

\end{document}

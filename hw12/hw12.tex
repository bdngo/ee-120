\documentclass{article}
\usepackage{eecstex}
\usepackage{physics}

\let\L\relax
\DeclareMathOperator{\L}{\mathcal{L}}
\let\Re\relax
\DeclareMathOperator{\Re}{\mathfrak{R}}

\renewcommand{\thesubsection}{\thesection.\arabic{subsection}}
\renewcommand{\thesubsubsection}{\thesubsection.\alph{subsubsection}}
\renewcommand{\labelenumi}{\arabic{enumi}.}
\newcommand{\F}{\mathcal{F}}
\newcommand{\sinc}{\operatorname{sinc}}
\newcommand{\rect}{\operatorname{rect}}


\title{EE 120 HW 12}
\author{Bryan Ngo}
\date{2021-04-22}

\begin{document}

\maketitle

\section{Laplace Transform}

\subsection{}

Using the frequency shift property,
\begin{equation}
    \L\{e^{-t} \cos(2t) u(t)\} = \frac{s + 1}{(s + 1)^2 + 4}
\end{equation}
and the ROC is \(\Re\{s + 1\} > 0\).

\subsection{}

\begin{align}
    \L\{\sin(t) u(-t)\} &= \int_{-\infty}^0 \sin(t) e^{-st} \, dt \\
    &= -\int_0^\infty \sin(t) e^{-st} \, dt \\
    &= -\int_0^\infty \sin(t) e^{-st} \, dt = -\frac{1}{s^2 + 1}
\end{align}
and the ROC is \(\Re\{s\} < 0\).

\subsection{}

Using the derivative property,
\begin{equation}
    \L\{t \cos(t) u(t)\} = -\dv{s} \L\{\cos(t) u(t)\} = -\dv{s} \frac{s}{s^2 + 1} = \frac{s^2 - 1}{(s^2 + 1)^2}
\end{equation}
and the ROC is unchanged, or \(\Re\{s\} > 0\).

\subsection{}

Using the time-shifting property,
\begin{equation}
    \L\{\delta(t - 1)\} = e^{-s}
\end{equation}
where the ROC is all of \(\R\).

\section{Inverse Laplace Transform}

\subsection{}

By the \(s\)-domain shifting property,
\begin{equation}
    \L^{-1}\qty{\frac{s + 2}{(s + 2)^2 + 1}} = e^{-2t} \cos(t) u(t)
\end{equation}

\subsection{}

By the time shifting property,
\begin{equation}
    \L^{-1}\qty{e^{-s} \frac{s}{s^2 + 1}} = \cos(t - 1) u(t - 1)
\end{equation}

\subsection{}

By the \(s\)-domain derivative property,
\begin{equation}
    \L^{-1}\qty{\frac{s}{(s^2 + 1)^2}} = -\frac{1}{2} \dv{s} \frac{1}{s^2 + 1} = \frac{1}{2} t \sin(t) u(t)
\end{equation}

\section{LTI System Transfer Function}

\begin{equation}
    H(s) = \frac{\omega_n^2}{s^2 + 2\zeta \omega_n s + \omega_n^2}
\end{equation}

\subsection{}

Finding the poles of \(H(s)\),
\begin{equation}
    s_p = -\zeta \omega_n \pm \sqrt{\zeta^2 \omega_n^2 - \omega_n^2} = -\zeta \omega_n \pm \omega_n \sqrt{\zeta^2 - 1} = \omega_n (-\zeta \pm \sqrt{\zeta^2 - 1}) = 0 \implies s_p = -\zeta \pm \sqrt{\zeta^2 - 1}
\end{equation}
In order to be stable, we must have \(\Re\{s_p\} < 0\).
If \(\zeta\) is in the desired interval, then the term in the square root will be imaginary and thus irrelevant, so \(\Re\{s\} = -\zeta\), which is always negative for our interval.

\subsection{}

When \(\zeta = 0\), then \(s_p = \pm j\omega_n\), so \(\Re\{s\} = 0\).

\subsection{}

Plugging in \(\zeta\) and \(\omega_n\), we have
\begin{equation}
    H(s) = \frac{1}{s^2 + 1}
\end{equation}
Then, consider the Laplace transform
\begin{equation}
    \frac{s}{(s^2 + 1)^2} \longleftrightarrow \frac{1}{2} t \sin(t) u(t)
\end{equation}
which is unbounded.
Then, we can find \(X(s)\) to be
\begin{equation}
    X(s) = \frac{Y(s)}{H(s)} = \frac{s}{s^2 + 1} \implies x(t) = \cos(t) u(t)
\end{equation}
which is bounded.

\subsection{}

Plugging in \(s = j \omega\),
\begin{align}
    H(\omega) &= \frac{\omega_n^2}{-\omega^2 + 2 \zeta \omega_n j \omega + \omega_n^2} \\
    |H(\omega)| &= \frac{\omega_n^2}{\sqrt{(\omega_n^2 - \omega^2)^2 + 4 \zeta^2 \omega_n^2 \omega^2}} \\
    &= \omega_n^2 (\omega_n^4 - 2\omega_n^2 \omega^2 + \omega^4 + 4 \zeta^2 \omega_n^2 \omega^2)^{-\frac{1}{2}} \\
    &= \omega_n^2 (\omega^4 + \omega_n^2 (4 \zeta^2 - 2) \omega^2 + \omega_n^4)^{-\frac{1}{2}}
\end{align}
Then, we can prove that \(|H(\omega)|\) is monotonically decreasing over the interval by finding the derivative of the term in the square root
This is because the square root function is monotonically increasing, which is monotonically \emph{decreasing} in the denominator,
\begin{equation}
    \dv{\omega} (\omega^4 + \omega_n^2 (4 \zeta^2 - 2) \omega^2 + \omega_n^4) = 4\omega^3 + 2\omega_n^2 (4\zeta^2 - 2) \omega
\end{equation}
Plugging in \(\zeta \geqslant \frac{1}{\sqrt{2}}\),
\begin{equation}
    4\omega^3 + 2\omega_n^2 (4\zeta^2 - 2) \omega \geqslant 4\omega^3 + 2\omega_n^2 \cancelto{0}{\qty(4 \qty(\frac{1}{\sqrt{2}})^2 - 2)} \omega = 4\omega^3
\end{equation}
Thus, our frequency response is monotonically decreasing for \(\omega \geqslant 0\).

\section{Transfer Function}

\subsection{}

Taking the Laplace transform on both sides of every equation yields
\begin{align}
    s Z_1(s) &= Z_2(s) \\
    s Z_2(s) &= -a_0 Z_1(s) - a_1 Z_2(s) + X(s) \\
    Y(s) &= b_0 Z_1(s) + b_1 Z_2(s) \\
\end{align}
We can substitute into \(Z_2(s)\) to be
\begin{align}
    s^2 Z_1(s) &= -a_0 Z_1(s) - a_1 s Z_1(s) + X(s) \\
    \implies X(s) &= Z_1(s) (s^2 + a_1 s + a_0) \\
    \implies H(s) &= \frac{Y(s)}{X(s)} = \frac{\cancel{Z_1(s)} (b_0 + b_1 s)}{\cancel{Z_1(s)} (s^2 + a_1 s + a_0)} = \frac{b_0 + b_1 s}{s^2 + a_1 s + a_0}
\end{align}

\subsection{}

\begin{align}
    s Z_1(s) &= -a_0 Z_2(s) + b_0 X(s) \\
    s Z_2(s) &= Z_1(s) - a_1 Z_2(s) + b_1 X(s) \\
    Y(s) &= Z_2(s)
\end{align}
Multiplying by \(s\) on both sides and substituting the first equation into the second,
\begin{align}
    s^2 Z_2(s) &= -a_0 Z_2(s) + b_0 X(s) - a_1 s Z_2(s) + b_1 s X(s) \\
    \implies X(s) &= Z_2(s) \frac{s^2 + a_1 s + a_0}{b_0 + b_1 s} \\
    \implies H(s) &= \frac{Y(s)}{X(s)} = \frac{b_0 + b_1 s}{s^2 + a_1 s + a_0}
\end{align}

\section{Second order LCCDE}

Finding the homogenous solution,
\begin{align}
    \lambda^2 + 3\lambda + 2 &= 0 \implies \lambda = -2, -1
    y_h(t) = c_0 e^{-2t} + c_1 e^{-t}
\end{align}
Finding the particular solution,
\begin{align}
    \dv[2]{y(t)}{t} + 3 \dv{y(t)}{t} + 2y(t) &= x(t) = e^{-3t} u(t) \\
    s^2 Y(s) + 3 s Y(s) + 2Y(s) &= \frac{1}{s + 3} \\
    \implies Y(s) &= \frac{1}{(s + 3) (s + 2) (s + 1)}
\end{align}
Then, by PFD,
\begin{align}
    \frac{1}{(s + 3) (s + 2) (s + 1)} &= \frac{A}{s + 3} + \frac{B}{s + 2} + \frac{C}{s + 1} \\
    \implies 1 &= A (s + 2) (s + 1) + B (s + 3) (s + 1) + C (s + 3) (s + 2) \\
    s = -1 &\implies C = \frac{1}{2} \\
    s = -2 &\implies B = -1 \\
    s = -3 &\implies A = \frac{1}{2} \\
    \implies Y(s) &= \frac{1}{2 (s + 3)} - \frac{1}{s + 2} + \frac{1}{2 (s + 1)} \\
    \implies y_p(t) &= u(t) \qty(\frac{1}{2} e^{-3t} - e^{-2t} + \frac{1}{2} e^{-t})
\end{align}
where the sidedness of the signal is assumed to be causal and thus right-handed.
Combining the two,
\begin{equation}
    y(t) = c_0 e^{-2t} + c_1 e^{-t} + u(t) \qty(\frac{1}{2} e^{-3t} - e^{-2t} + \frac{1}{2} e^{-t})
\end{equation}
Since this is a causal system, we know that \(y(0) = \dv{t} y(0) = 0\).
Plugging in,
\begin{align}
    0 &= c_0 + c_1 \\
    0 &= -2 c_0 - c_1
\end{align}
which has the unique solution \(c_0 = c_1 = 0\), so our final system is
\begin{equation}
    y(t) = u(t) \qty(\frac{1}{2} e^{-3t} - e^{-2t} + \frac{1}{2} e^{-t})
\end{equation}

\end{document}
